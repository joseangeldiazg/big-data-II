%---------------------------------------------------
% Nombre: capitulo1.tex  
% 
% Texto del capitulo 1
%---------------------------------------------------

\chapter{Introducci�n}

En este documento encontramos el resultado final de la pr�ctica \textbf{ETL con PIG} \cite{pig} , enmarcada dentro de la asignatura de Big Data II del m�ster en Ciencia de Datos de la Universidad de Granada. 

\section{Problema a resolver}
\label{problema}

Se pide dise�ar un experimento de carga y extracci�n de informaci�n de una base de datos usando \textbf{PIG}, para realizar consultas en un entorno distribuido Hadoop. PIG es un lenguaje creado para flujo de datos que permite especificar como captar y leer estos flujos. Es �til para:

\begin{itemize}
\item ETL.
\item Analizar datos en bruto.
\item Procesamiento de datos iterativo.
\end{itemize}

La base de datos elegida, es una base de datos de quejas de usuarios de tarjeta de cr�dito en Estados Unidos, con un total de 65499 muestras con 18 caracter�sticas por cada muestra. 

\section{Objetivos}

Los objetivos de esta pr�ctica podr�an resumirse en los siguientes:

\begin{itemize}
	\item Carga de los datos.
	\item Realizar un experimento original.
	\item Resolver el experimento con consultas que incluyan una proyecci�n, selecci�n, agrupamiento y funciones de agregado y resumen sobre los datos.
\end{itemize}

\section{Organizaci�n del trabajo} 
Tras la introducci�n al problema y los objetivos de la pr�ctica, en el cap�tulo \ref{dos} detallamos cada uno de los pasos seguidos para el dise�o del experimento de datos, junto con las salidas ofrecidas por cada consulta. En el cap�tulo \ref{tres} analizamos las conclusiones obtenidas durante el transcurso de la pr�ctica. 

\clearpage
%---------------------------------------------------