%?????????????????????????
% Nombre: capitulo5.tex  
% 
% Texto del capitulo 5
%---------------------------------------------------

\chapter{Conclusi�n}
\label{tres}

Tras la realizaci�n de la pr�ctica y su comparaci�n con la practica realizada anteriormente de ETL en sistemas distribuidos con IMPALA, cabe destacar que la curva de aprendizaje en PIG es mayor que en Impala, debido a que este �ltimo usa la sem�ntica de SQL, lo que facilita enormemente las cosas. En cuanto a velocidad destacar tambi�n que se nota que PIG es un motor m�s lento y que no est� pensado para consultas b�sicas como las realizadas en la practica sino para problemas de flujo de datos potentes donde realmente podamos beneficiarnos del tiempo de gesti�n de los procesos map-reduce. 

Por otro lado, queda constancia de que es un sistema bastante interesante y vers�til por lo que de cara a la profesi�n de analista de datos en Big Data conocer  herramientas de este tipo puede marcar la diferencia.

\pagebreak
\clearpage
%---------------------------------------------------